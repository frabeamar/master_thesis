% set counter to n-1:
\setcounter{chapter}{0}

\chapter{Introduction}

The goal of human pose estimation is to localize semantic keypoints of a human body. 
Recently, many methods utilize deep convolutional neural networks (CNNs) a
Conventionally, the pose refinement has been mainly performed by multi-stage architectures 


Existing methods for 3D motion capture of athletes can
be classified into those that use motion sensors and those
that use visual information only. The methods that use motion sensors can detect the accurate movement of the human
skeleton by attaching markers to the human body. However, dedicated equipment such as motion sensors are expensive and the environment for measurement is limited.
 We refine OpenPose with our newly introduce sports
dataset and a augmentation technique to improve the quality of
2D pose estimation in extreme poses.
In addition, markers must be attached on the human body,
which requires expert knowledge and may not be comfortable for the athlete. As a consequence, non-invasive techniques such as 3D pose estimation from multi-view RGB
images are preferred ( \cite{DBLP:journals/corr/abs-1907-11346}, \cite{DBLP:journals/corr/abs-1904-01324} , 
\cite{Xu_2020_CVPR}, \cite{DBLP:journals/corr/abs-2003-00529}, \cite{DBLP:journals/corr/abs-2006-07778}

[ 29],[13]). In general,
existing methods first estimate the 2D pose of the person
in each 2D image and then triangulate the 2D skeletons to
create a 3D skeleton. By doing so, a marker-less, low-cost
system that can be used anywhere can be built. In such systems, accurate and robust 2D pose estimation is critical.

If taken into account the ultimate goal of using such systems in video production the attention 

s. First, collecting accurate 3D pose annotation for RGB
images is expensive and time-consuming

The first
stage locates 2D human key-points from appearance information, while the second stage lifts the 2D joints into 3D
skeleton employing geometric information. S